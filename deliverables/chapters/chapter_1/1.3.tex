\section{Calibration matrix}

The goal of this section is to compute the calibration matrix $\mathbf{K}$, which encodes the intrinsic parameters of the camera. 
The calibration matrix is given by:
\[\mathbf{K} = \begin{bmatrix} f_x & s & u_0 \\ 0 & f_y & v_0 \\ 0 & 0 & 1 \end{bmatrix}\]
Here: 
\begin{itemize}
    \item $f_x$ and $f_y$ are the effective focal lengths along the $x$ and $y$-axes, respectively.
    \item $u_0$ and $v_0$ are the coordinates of the principal point (the optical center in the image).
    \item $s$ is the skew parameter, which describes the angle between the image axes. 
        By assumption, we are using a zero-skew camera, so $s=0$. 
\end{itemize}

\subsection{Image of the Absolute Conic}
To estimate the calibration matrix, we leverage the Image of the Absolute Conic (IAC), represented by the symmetric matrix:
\[\boldsymbol{\omega} = \begin{bmatrix} a & 0 & b \\ 0 & 1 & c \\ b & c & d \end{bmatrix}\]
The IAC encodes intrinsic camera parameters and can be used to recover the elements of $\mathbf{K}$. 

To determine $\boldsymbol{\omega}$, we impose geometric constraints based on vanishing points and the line at infinity.
We will use the following elements fo the contraints: 
\begin{enumerate}
    \item \textit{Vertical vanishing point}: $\mathbf{ph}$.
    \item \textit{Homography matrix}: $\mathbf{H}=(\mathbf{H}_{\text{metric}}\mathbf{H}_{\text{rect}})^{-1}$, which combines Euclidean and metric transformations. 
\end{enumerate}
The constraints imposed on $\boldsymbol{\omega}$ are as follows: 
\[\begin{cases}
    \mathbf{ph}^T\boldsymbol{\omega}\mathbf{h}_1 = 0 \\
    \mathbf{ph}^T\boldsymbol{\omega}\mathbf{h}_2 = 0  \\
    \mathbf{h}_1^T \boldsymbol{\omega} \mathbf{h}_2 = 0 \\
    \mathbf{h}_1^T \boldsymbol{\omega} \mathbf{h}_1 - \mathbf{h}_2^T \boldsymbol{\omega} \mathbf{h}_2 = 0 \\
\end{cases}\]
Here $\mathbf{h}_1$ and $\mathbf{h}_2$ are the first and second columns of the homography matrix $\mathbf{H}$, respectively. 

The constraints form a linear system of equations that can be solved to recover the coefficients $a$, $b$, $c$, and $d$ of the IAC matrix $\boldsymbol{\omega}$. 
Once $\boldsymbol{\omega}$ is determined, the intrinsic parameters of $\mathbf{K}$ can be computed.

\subsection{Intrinsic parameters estimation}
The intrinsic parameters are computed as follows:
\[\begin{cases}
    f_x = \sqrt{d - \alpha^2 \cdot u_0^2 - v_0^2} \\
    f_y = \frac{f_y}{\alpha} \\
    u_0 = - \frac{b}{\alpha^2} \\
    v_0 = - c
\end{cases}\]
Here, $\alpha = \sqrt{a}$. 